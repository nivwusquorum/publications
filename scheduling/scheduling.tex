\documentclass{article}
\usepackage[utf8]{inputenc}


\usepackage{natbib}
\usepackage{amsfonts}
\usepackage{mathtools}
\usepackage{graphicx}
\usepackage{multicol}
\usepackage{blindtext}
\usepackage[top=2cm, bottom=2cm, left=2cm, right=2cm]{geometry}
\usepackage{fancyhdr}

\bibliographystyle{apalike}


%%%%%%%%%%%%%%%%%%%%%%%%%%%%%% FANCY HEADER %%%%%%%%%%%%%%%%%%%%%%%%%%%%%%%%%%%
\fancyhead{}
\fancyfoot{}
\fancyhead[C]{\bf Strong controllability with uncertainty in resource and time.}
\renewcommand{\headrulewidth}{1pt}
\newcommand{\HRule}{\rule{\linewidth}{0.5mm}}

%%%%%%%%%%%%%%%%%%%%%%%%%%%%%% DOCUMENT %%%%%%%%%%%%%%%%%%%%%%%%%%%%%%%%%%%%%%%

\begin{document}
\thispagestyle{empty}

\begin{center}
%%%%%%%%%%%%%%%%%%%%%%%%%%%%%% TITLE PAGE %%%%%%%%%%%%%%%%%%%%%%%%%%%%%%%%%%%%%
\HRule \\[0.3cm]
{\Large \bfseries Strong controllability with uncertainty in resource and time. \\[0.3cm]}
\HRule \\[0.5cm]

\noindent
\begin{minipage}{0.5\textwidth}
\begin{flushleft}
\textbf{Szymon Sidor\\
Peng Yu\\
Cheng Fang\\
Brian Williams}\\
Massachusetts Institute of Technology
\end{flushleft}
\end{minipage}%
\begin{minipage}{0.5\textwidth}
\begin{flushright}
\textsc{sidor@mit.edu}\\
\textsc{yupeng@mit.edu}\\
\textsc{cfang@mit.edu}\\
\textsc{derek.aylward@gmail.com}\\
$\ $
\end{flushright}
\end{minipage}
\\[1cm]
\end{center}
\pagestyle{fancy}

\begin{multicols}{2}
%%%%%%%%%%%%%%%%%%%%%%%%%%%%%% ABSTRACT %%%%%%%%%%%%%%%%%%%%%%%%%%%%%%%%%%%%%%%
\begin{abstract}
\noindent \blindtext
\end{abstract}
%%%%%%%%%%%%%%%%%%%%%%%%%%%%%% INTRODUCTION %%%%%%%%%%%%%%%%%%%%%%%%%%%%%%%%%%%
\section{Introduction}
\blindtext[5]

%%%%%%%%%%%%%%%%%%%%%%%%%%%%%% RELATED WORK %%%%%%%%%%%%%%%%%%%%%%%%%%%%%%%%%%%
\section{Related Work}
\blindtext[5]

%%%%%%%%%%%%%%%%%%%%%%%%%%%%%% PROBLEM STATEMENT %%%%%%%%%%%%%%%%%%%%%%%%%%%%%%
\section{Problem statement}
In this section we will define notion of a Time Resource Network. For such a network we will two properties $p$-Strong Controllability and Resource Consistency. The problem we wish to determine is finding a schedule that satisfies both of those properties.
\subsection{Time Resource Network}
Let's first define general problem of scheduling with resource limits. We will extend the notion of a Chance Constrained Probabilistic Simple Temporal Network defined by \cite{Fang2014}. Time Resource Network defines two types of constraints - time constraints and resource constraints. Time constraints are expressed by the following variables:
\begin{itemize}
\item Set of abstract time points $V = V_C \cup V_R $.
\begin{itemize}
\item $V_C$ is a set of controllable time points (chosen by us)
\item $V_R$ is a set of random time points (chosen by environment)
\end{itemize}
\item Set of time constraints $E = E_L \cup E_P $.
\begin{itemize}
\item $E_L$ is a set of linear time constraints of form $(y-x) \in [l,u]$, where $x,y \in V$ and $l,u \in \mathbb{R}$.
\item $E_P$ is a set of probabilistic time constraints of form $ (y-x) \sim D$ where $x \in V_C$, $y \in V_R$ and $D$ is some probability distribution.
\end{itemize}
\item an upper bound on total probability of failure $p_f$ (acceptable risk)
\end{itemize}
Additionally we require that each random time point is and end (a $y$ variable) of exactly one probabilistic constraint and vice versa (every probabilistic constraint begins with controllable time point and ends with a random time point).\\
Resource constraints are expressed by the following variables:
\begin{itemize}
\item Set of resources $R$
\item Resource usage/production edges $R_E$ which are of form $x \xrightarrow{r:c} y$ where $r\in R$ and $c \in \mathbb{R}$. If $c$ is positive this means that on the time interval $[x,y]$ there are $c$ units of resource $r$ available. Similarly if $c$ is negative it means that on the time interval $[x,y]$, the amount of $c$ units of resource $r$ is used.
\end{itemize}
We do not allow to define resource usage for random time points.
Given a TRN $(V,E,r,R,R_E)$ we can define a \textbf{schedule} $s: V_c \rightarrow \mathbb{R}$, which is a function assigning execution time to controllable time points.
\subsection{$p$-Strong Controllability}
\cite{Fang2014} extended notion of strong controllability to pSTNs, which naturally carries on to TRN.  For a given schedule $s$ and assignment $a: V_R \rightarrow \mathbb{R}$ to random time points we can define probability that assignment
\begin{align*}
Pr(a) = \prod_{(y-x \sim D) \in E_P} Pr_D(a(y)-s(x))
\end{align*}.
Moreover we say assignment $a$ is consistent under schedule $s$ if it satisfies all the linear time constraints in $E_L$.
We say that probability network is \textbf{$p$-strongly controllable} if there exists a schedule $s$ such that with probability at least $p$ the assignment $a$ is consistent under schedule $s$ i.e.
\begin{align*}
\exists_s \int_{a} Pr(a) [\text{s,a satisfy all the linear constraints} ] \geq p
\end{align*}

\subsection{Resource consistency}
For a given schedule $s$ let's define $U_r^s(t)$ as the usage of resource $r$ at time $t$ under schedule $s$. More formally
\[
U_r(t) = \sum_{\forall_{x \xrightarrow{r:c} y \in R_E}. s(x)\leq t \leq s(y)} c
\]
We say the network satisfy the network is \textbf{resource consistent} under schedule $s$ if it newer uses more resources than limit, i.e.
\[
\forall_{r \in R} \forall_{t} . U_r^s(t) \leq 0
\]
\subsection{The problem}
The problem that we are solving in this paper is determining whether TRN is consistent i.e. whether
there exists a schedule such that under that schedule network is $(1-p_f)$-strongly consistent and resource consistent.


%%%%%%%%%%%%%%%%%%%%%%%%%%%%%% ALGORITHM %%%%%%%%%%%%%%%%%%%%%%%%%%%%%%%%%%%%%%
\section{Algorithm}
\blindtext[5]
%%%%%%%%%%%%%%%%%%%%%%%%%%%%%% EXPERIMENTS %%%%%%%%%%%%%%%%%%%%%%%%%%%%%%%%%%%%
\section{Experiments}
\blindtext[5]
%%%%%%%%%%%%%%%%%%%%%%%%%%%%%% FUTURE WORK %%%%%%%%%%%%%%%%%%%%%%%%%%%%%%%%%%%%
\section{Future Work}
\subsection{Conflict directed scheduling with resource limits(Peng and Szymon)}
For a network which there does not exist an assignment to controllable time points that is consistent with both time and resource constraints we wish to provide find a conflict - a suggested change to problem formulation which bring it closer to feasible solution.
\subsection{Possible extensions}
Below are possible extensions. I think the first one is simple to solve. The 2nd and 3rd are potentially hard. The last one sounds extremely hard to solve in reasonable execution time even for small problems.
\subsubsection{time-varying load profile}
Variable load profile can be approximated by a piecewise constant function with a series of temporal constraints and resource usage information. Does \textbf{not} require augmenting problem definition
\subsubsection{resource usage usage defined on random time points}
Requires augmenting problem definition.
\subsubsection{time-varying resource limit}
Could also potentially be approximated by piecewise constant function. Requires augmenting problem definition.
\subsubsection{stochastic resource limit}
Could potentially simulate multiple high likelihood scenarios to estimate probability of failure. Requires augmenting problem definition.%%%%%%%%%%%%%%%%%%%%%%%%%%%%%% CONCLUSION %%$$$%%%%%%%%%%%%%%%%%%%%%%%%%%%%%%%%
\section{Conclusion}
\blindtext[5]
%%%%%%%%%%%%%%%%%%%%%%%%%%%%%% REFERENCES %%%%%%%%%%%%%%%%%%%%%%%%%%%%%%%%%%%%%
\section{References}

\bibliography{references}
\end{multicols}
\end{document}
