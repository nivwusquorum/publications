\documentclass{article}
\usepackage[utf8]{inputenc}


\usepackage[lined, boxed, linesnumbered]{algorithm2e}
\usepackage{natbib}
\usepackage{amsfonts}
\usepackage{amsthm}
\usepackage{mathtools}
\usepackage{graphicx}
\usepackage{multicol}
\usepackage{blindtext}
\usepackage[top=2cm, bottom=2cm, left=2cm, right=2cm]{geometry}
\usepackage{fancyhdr}

\bibliographystyle{apalike}

\newtheorem{theorem}{Theorem}[section]
\newtheorem{corollary}{Corollary}[theorem]
\newtheorem{lemma}[theorem]{Lemma}


%%%%%%%%%%%%%%%%%%%%%%%%%%%%%% FANCY HEADER %%%%%%%%%%%%%%%%%%%%%%%%%%%%%%%%%%%
\fancyhead{}
\fancyfoot{}
\fancyhead[C]{\bf Extending Temporal Networks with Probabilistic Resource Constraints.}
\renewcommand{\headrulewidth}{1pt}
\newcommand{\HRule}{\rule{\linewidth}{0.5mm}}

%%%%%%%%%%%%%%%%%%%%%%%%%%%%%% DOCUMENT %%%%%%%%%%%%%%%%%%%%%%%%%%%%%%%%%%%%%%%

\begin{document}
\thispagestyle{empty}

\begin{center}
%%%%%%%%%%%%%%%%%%%%%%%%%%%%%% TITLE PAGE %%%%%%%%%%%%%%%%%%%%%%%%%%%%%%%%%%%%%
\HRule \\[0.3cm]
{\Large \bfseries Extending Temporal Networks with Probabilistic Resource Constraints \\[0.3cm]}
\HRule \\[0.5cm]

\noindent
\begin{minipage}{0.5\textwidth}
\begin{flushleft}
\textbf{Szymon Sidor\\
Peng Yu\\
Cheng Fang\\
Brian Williams}\\
Massachusetts Institute of Technology
\end{flushleft}
\end{minipage}%
\begin{minipage}{0.5\textwidth}
\begin{flushright}
\textsc{sidor@mit.edu}\\
\textsc{yupeng@mit.edu}\\
\textsc{cfang@mit.edu}\\
\textsc{derek.aylward@gmail.com}\\
$\ $
\end{flushright}
\end{minipage}
\\[1cm]
\end{center}
\pagestyle{fancy}

\begin{multicols}{2}
%%%%%%%%%%%%%%%%%%%%%%%%%%%%%% ABSTRACT %%%%%%%%%%%%%%%%%%%%%%%%%%%%%%%%%%%%%%%
\begin{abstract}
\noindent %\blindtext
\end{abstract}
%%%%%%%%%%%%%%%%%%%%%%%%%%%%%% INTRODUCTION %%%%%%%%%%%%%%%%%%%%%%%%%%%%%%%%%%%
\section{Introduction}
%\blindtext[5]

%%%%%%%%%%%%%%%%%%%%%%%%%%%%%% RELATED WORK %%%%%%%%%%%%%%%%%%%%%%%%%%%%%%%%%%%
\section{Related Work}
%\blindtext[5]

%%%%%%%%%%%%%%%%%%%%%%%%%%%%%% PROBLEM STATEMENT %%%%%%%%%%%%%%%%%%%%%%%%%%%%%%
\section{Problem statement}
In this section we will define notion of a Time Resource Network (TRN) and the relevant constraint on TRN's schedule - Resource Consistency. All the results presented in this paper scale to multiple different type of resources being constraint at the same time (electricity, water, fuel, cpu time, memory etc.), but to simplify the notation we will assume that only one type of resource is constrained.
\subsection{Abstract Temporal Network}
Since TRNs can operate on top of many different types of temporal networks, we define a notion of Abstract Temporal Network (ATN), to capture only the necessary properties. For abstract temporal network we define two pieces of functionality:
\begin{enumerate}
\item \texttt{nodes(ATN)}, which returns a set of timepoints in $ATN$
\item \texttt{extend(ATN, $\{ stc_1, ... stc_n \} $)}, which takes ATN and set of simple temporal constraints \cite{cervoni1993maintaining} spanning \texttt{nodes(ATN)}, and returns another $ATN'$, such that there exists a schedule satisfying $p-TC(ATN')$ if and only if there exists a schedule satisfying $p-TC(ATN)$ and the obeying set of simple temporal constraint $\{ stc_1, ... stc_n \} $. $p-TC$ is a notion of probabilistic temporal consistency described in section \ref{temporal_consistency}.
\end{enumerate}
As the following section describes in detail we will use \texttt{extend} to encode resource constraints over \texttt{nodes}.
\subsection{Temporal Consistency}
\label{temporal_consistency}
For an ATN we define a predicate $TC(ATN)$. $TC$ is true if we can find acceptable execution strategy for that network (what that means precisely depends on the ATN - we only require for it to be verifiable).
\paragraph{Example} Let's consider cc-pSTP \cite{Fang2014} as an example. Here \texttt{nodes} returns set of \textit{activated} and \textit{received} timepoints. \texttt{extend} returns network with extra \textit{free contraints} encoding the simple temporal constraints. The temporal consistency check $TC$ is true if cc-pSTP has a solution.
\subsection{Time Resource Network}
A Time Resource Network $TRN = (ATN, {rc_1, ..., rc_n})$, where $ATN$ is an Abstract Temporal Network and ${rc_1, ..., rc_n}$ is a set of resource constraints. There are a few different types of resource constraints, but they all have a common property - they describe resource usage between two timepoints in \texttt{nodes(ATN)}.
\subsection{Schedule}
A schedule $s: \texttt{nodes(ATN)} \rightarrow \mathbb{R}$ is a mapping from abstract timepoints in ATN to concrete execution times.
\subsection{Resource constraints}
\begin{itemize}
\item \textbf{simple resource constraint} is a triplet $(x, y, r)$, where $x, y \in$ \texttt{nodes(ATN)} and $r$ is resource usage which can be positive (consumption) and negative (generation). For any time $s(x) \leq t \leq s(y)$ usage is equal to
\[
u(t) = r
\]
Where $s$ is a schedule. Intuitively, simple resource constraint encodes the fact that between time $s(x)$ and $s(y)$  resource is consumed (generated) at the rate $|r$| units of resource per unit time for positive (negative) $r$.
\item \textbf{linear resource constraint} is a triplet $(x, y, r_b, r_e)$, where $x, y \in \texttt{nodes(ATN)}$ and resource usage at time $s(x) \leq t \leq s(y)$ is equal to
\[
    u(t) = r_b + t  \frac{r_e - r_b}{s(y) - s(x)}
\]
Intuitively, simple resource constraint encodes the fact that between time $s(x)$ and $s(y)$  resource is consumed/generated with rate that changes linearly between $s(x)$ and $s(y)$.
\item \textbf{probabilistic simple resource constraint}
Is an extension of simple resource constraint where $r$ is a random variable (and therefore so is $u(t)$).
\item \textbf{probabilistic linear resource constraint}
Is an extension of simple resource constraint where $r_b$ and $r_e$ are a random variables (and therefore so is $u(t)$).
\end{itemize}
Notice in particular that all the resource constraints are special cases of probabilistic linear resource constraint. Indeed, constant functions family is a subset of linear function family and the non-probabilistic versions of constraints can be though of as Dirac delta function probability distributions.

\paragraph{Example} Describe solar panel example.


\subsection{$p$-Resource consistency}
For a given schedule $s$ let's define random variable $U_s(t)$ as net usage of the resource at time $t$ under schedule $s$. More formally
\[
U_s(t) = \sum_{\forall_{(x_i,y_i,u_i) \in R}. s(x_i)\leq t \leq s(y_i)} u_i(t)
\]
Where $R$ is a set of all the resource constraints and $x_i$, $y_i$ and $u_i$ are as described in section about linear resource constraints ($r_b$ and $r_e$ are encapsulated by $u$). We say the network satisfy the network is \textbf{$p$-resource consistent} under schedule $s$ if it newer uses more resources than limit, i.e.
\[
\forall_{r \in R} \forall_{t} . U_s(t) \leq 0
\]
with probability $p$. When dealing with the special case of non-probabilistic duration we drop the $p$ and talk about \textbf{resource-consistency} as for any non-zero value of $p$ the outcome is the same.

\subsection{Time-resource consistency}
The problem that we are solving in this paper is determining whether TRN is time-resource consistent. Before defining exactly what this means we prove the following lemma:

\begin{lemma}
\label{resource_checking}
Assume we have a TRN, such that ATN has explicit timepoints for start and end of the time horizon (can be $\pm \infty$).
A schedule is resource-consistent if and only if
\[
\forall_{t \in \texttt{nodes(ATN)}} U_s(s(t)) \leq 0
\]
i.e. resource usage is not non-positive at all of the scheduled timepoints.
\end{lemma}
\begin{proof}
$\Rightarrow$ Trivial from definition of resource-consistency.
$\Leftarrow$ We say a time $t \in \mathbb{R}$ is scheduled if there exists a timepoint  $x \in \texttt{nodes(ATN)}$ such that $t = s(x)$. We can rephrase right side of the lemma saying that for all the scheduled $t$ $U_s(t) \leq 0$. Assume that the right side of the implication is satisfied but the schedule is not resource consistent. That means that there exists a time point $t_{danger}$ which is not schduled for which $U_s(t_{danger}) > 0 $. Let $x$ be highest scheduled time less than $t_{danger}$ and let $y$ be smallest scheduled time higher than $t_{danger}$. We know that $U_s(t)$ for $x \leq t \leq y$ is a sum of linear functions (or possibly zero), which is itself a linear function. This means that either $U_s(x) \geq t_{danger}$ or $U_s(y) \geq t_{danger}$, so either $U_s(x) > 0$ or $U_s(y) > 0$. But both $x$ and $y$ are scheduled. Contradiction.

\end{proof}
\begin{corollary}
Given a $TRN$ with only simple resource constraints and two schedules $A$ and $B$ that have the same ordering of timepoints, $A$ is $p$-resource-consistent if and only if $B$ is $p$-resource-consistent.
\end{corollary}
\begin{proof}
Notice that if we move arbitrary timepoint, while preserving the relative ordering of timepoints, then net resource usage at that timepoint will not change (as the $U_s(t)$ between the neighboring timepoints remains constant). Therefore by lemma \ref{resource_checking} we can transform schedule $A$ into schedule $B$.
\end{proof}
Notice that the Corollary does not apply to the linear resource constraints (see fig. TODO).

TODO: generalize to probabilistic

Finally we say that a TRN is \textbf{$p$-time-resource consistent} if there exists and ordering of timepoints such that every schedule that satisfies this ordering is $p$-resource-consistent and $ATN$ extended with that ordering is $TC$.

%%%%%%%%%%%%%%%%%%%%%%%%%%%%%% ALGORITHM %%%%%%%%%%%%%%%%%%%%%%%%%%%%%%%%%%%%%%
\section{Algorithm}
High level idea of the algorithm is quite simple and is presented in algorithm \ref{hl_algo}. In the second line we iterate over all the permutations of the timepoints. On line 3 we use \texttt{p\_resource\_consistent} function to check resource consistency, performing this check is the nontrivial part of the algorithm. On line four we use $TC$ checker to determine if network is time consistent - the implementation depends on $ATN$ and we assume it is available. Function $encode\_as\_scts$ encodes permutation using simple temporal constraints. For example if $\sigma(1) = 2$ and $\sigma(2) = 1$ and $\sigma(3) = 3$, then we can encode it by two STCs: $ 2 \leftarrow 1 $ and $1 \leftarrow 3$.

\begin{algorithm}[H]
    \label{hl_algo}
    \KwData{TRN and p}
    \KwResult{true if TRN=(ATN, RC) is p-time-resource-consistent}
    $N \leftarrow \texttt{nodes(ATN)}$\;
    \For{$\sigma \leftarrow \text{permutation of } N$}{
        \If{\texttt{p\_resource\_consistent(RC, $\sigma$, $p$)} }{
            \If{TC(\texttt{extend(ATN, encode\_as\_scts($\sigma$))})}{
                succeed\;
            }
        }
    }
    fail\;
    \caption{Checking $p$-time-resource-consistency of a TRN }
\end{algorithm}
Implementation of \texttt{p\_resource\_consistent} follows from lemma \ref{resource_checking}.



% INPUT: atn N, {src} S (spanning N.timepoints)
% OUTPUT: scheduling strategy on N or fail
% ALGORITHM:
% X = subset of N.timepoints used by SRCs from S
% for every permutation pi of X:
% stcs = pi encoded by STCs
% result = N.solve_with_stcs(stcs)
% if result is schedule:
%        return schedule
%       fail



%%%%%%%%%%%%%%%%%%%%%%%%%%%%%% EXPERIMENTS %%%%%%%%%%%%%%%%%%%%%%%%%%%%%%%%%%%%
\section{Experiments}
%\blindtext[5]
%%%%%%%%%%%%%%%%%%%%%%%%%%%%%% FUTURE WORK %%%%%%%%%%%%%%%%%%%%%%%%%%%%%%%%%%%%
\section{Future Work}
%%%%%%%%%%%%%%%%%%%%%%%%%%%%%% CONCLUSION %%%%%%%%%%%%%%%%%%%%%%%%%%%%%%%%%%%%%
\section{Conclusion}
%\blindtext[5]
%%%%%%%%%%%%%%%%%%%%%%%%%%%%%% REFERENCES %%%%%%%%%%%%%%%%%%%%%%%%%%%%%%%%%%%%%
\section{References}

\bibliography{references}
\end{multicols}
\end{document}
