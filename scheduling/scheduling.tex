\documentclass{article}
\usepackage[utf8]{inputenc}


\usepackage{natbib}
\usepackage{amsfonts}
\usepackage{mathtools}
\usepackage{graphicx}
\usepackage{multicol}
\usepackage{blindtext}
\usepackage[top=2cm, bottom=2cm, left=2cm, right=2cm]{geometry}
\usepackage{fancyhdr}

\bibliographystyle{apalike}


%%%%%%%%%%%%%%%%%%%%%%%%%%%%%% FANCY HEADER %%%%%%%%%%%%%%%%%%%%%%%%%%%%%%%%%%%
\fancyhead{}
\fancyfoot{}
\fancyhead[C]{\bf Strong controllability with uncertainty in resource and time.}
\renewcommand{\headrulewidth}{1pt}
\newcommand{\HRule}{\rule{\linewidth}{0.5mm}}

%%%%%%%%%%%%%%%%%%%%%%%%%%%%%% DOCUMENT %%%%%%%%%%%%%%%%%%%%%%%%%%%%%%%%%%%%%%%

\begin{document}
\thispagestyle{empty}

\begin{center}
%%%%%%%%%%%%%%%%%%%%%%%%%%%%%% TITLE PAGE %%%%%%%%%%%%%%%%%%%%%%%%%%%%%%%%%%%%%
\HRule \\[0.3cm]
{\Large \bfseries Strong controllability with uncertainty in resource and time. \\[0.3cm]}
\HRule \\[0.5cm]

\noindent
\begin{minipage}{0.5\textwidth}
\begin{flushleft}
\textbf{Szymon Sidor\\
Peng Yu\\
Cheng Fang\\
Brian Williams}\\
Massachusetts Institute of Technology
\end{flushleft}
\end{minipage}%
\begin{minipage}{0.5\textwidth}
\begin{flushright}
\textsc{sidor@mit.edu}\\
\textsc{yupeng@mit.edu}\\
\textsc{cfang@mit.edu}\\
\textsc{derek.aylward@gmail.com}\\
$\ $
\end{flushright}
\end{minipage}
\\[1cm]
\end{center}
\pagestyle{fancy}

\begin{multicols}{2}
%%%%%%%%%%%%%%%%%%%%%%%%%%%%%% ABSTRACT %%%%%%%%%%%%%%%%%%%%%%%%%%%%%%%%%%%%%%%
\begin{abstract}
\noindent \blindtext
\end{abstract}
%%%%%%%%%%%%%%%%%%%%%%%%%%%%%% INTRODUCTION %%%%%%%%%%%%%%%%%%%%%%%%%%%%%%%%%%%
\section{Introduction}
\blindtext[5]

%%%%%%%%%%%%%%%%%%%%%%%%%%%%%% RELATED WORK %%%%%%%%%%%%%%%%%%%%%%%%%%%%%%%%%%%
\section{Related Work}
\blindtext[5]

%%%%%%%%%%%%%%%%%%%%%%%%%%%%%% PROBLEM STATEMENT %%%%%%%%%%%%%%%%%%%%%%%%%%%%%%
\section{Problem statement}
In this section we will define notion of a Time Resource Network. For such a network we will two properties $p$-Strong Controllability and Resource Consistency. The problem we wish to determine is finding a schedule that satisfies both of those properties.
\subsection{Time Resource Network}
Let's first define general problem of scheduling with resource limits. We will extend the notion of a Chance Constrained Probabilistic Simple Temporal Network defined by \cite{Fang2014}. Time Resource Network defines two types of constraints - time constraints and resource constraints. Time constraints are expressed by the following variables:
\begin{itemize}
\item Set of abstract time points $V = V_C \cup V_R $.
    \begin{itemize}
    \item $V_C$ is a set of controllable time points (chosen by us)
    \item $V_R$ is a set of random time points (chosen by environment)
    \end{itemize}
\item Set of time constraints $E = E_L \cup E_P $.
    \begin{itemize}
    \item $E_L$ is a set of linear time constraints of form $(y-x) \in [l,u]$, where $x,y \in V$ and $l,u \in \mathbb{R}$.
    \item $E_P$ is a set of probabilistic time constraints of form $ (y-x) \sim D$ where $x \in V_C$, $y \in V_R$ and $D$ is some probability distribution. Additionally we require that each random time point is and end (a $y$ variable) of exactly one probabilistic constraint and vice versa. We can think of a probabilistic constraint as defining constraint for a random time point.
    \end{itemize}
\end{itemize}
In this paper we define network consuming single type of resource but extension to many resources is straightforward. Constraints are expressed by the following variables:
\begin{itemize}
\item A set of resource constraints $E_R$, each of which is of form $x \xrightarrow{D} y$ where $x,y \in V_C$ and $D$ is some distribution on resource usage. We allow positive as well as negative usage; the former means consumption rate and the latter means production rate between time points $x$ and $y$. Notice that for the purpose of this publication we do not allow to define resource usage for random time points.
\end{itemize}

\noindent Given a TRN $(V,E,r,R,R_E)$ we can define a \textbf{schedule} $s: V_c \rightarrow \mathbb{R}$, which is a function assigning execution time to controllable time points.
\subsection{$p$-Strong Controllability}
\cite{Fang2014} extended notion of strong controllability to pSTNs, which naturally carries on to TRN.  For a given schedule $s$ and assignment $a: V_R \rightarrow \mathbb{R}$ to random time points we can define probability of that assignment as
\begin{align*}
Pr(a) = \prod_{(y-x \sim D) \in E_P} Pr_D(a(y)-s(x))
\end{align*}.
Moreover we say assignment $a$ is \textbf{consistent} under schedule $s$ if it satisfies all the linear time constraints in $E_L$ i.e.
\begin{align*}
\forall_{(y-x \in [l,u]) \in E_L} l \leq A(y) - A(x) \leq u
\end{align*}
Where $A$ is defined as
\begin{align*}
A(x)=\begin{cases}
s(x) & if\ x \in V_C\\
a(x) & if\ x \in V_R
\end{cases}
\end{align*}
We say that probability network is \textbf{$p$-strongly controllable} under schedule $s$ if with probability $p$ assignment $a$ is consistent under $s$.

\subsection{$p$-Resource consistency}
For a TRN we define resource usage as $u:E_R \rightarrow \mathbb{R}$. For resource usage we define its probability as a product of probabilities of every resource constraint taking that value (according to its distribution $D$).
For a given schedule $s$ and usage $u$ let's define $U_u^s(t)$ as net usage of a resource at time $t$ under schedule $s$. More formally
\[
U_u^s(t) = \sum_{\forall_{c \in R_E}. s(x)\leq t \leq s(y)} u(c)
\]
We say the network satisfy the network is \textbf{resource consistent} under schedule $s$ and usage $u$ if it newer uses more resources than limit, i.e.
\[
\forall_{r \in R} \forall_{t} . U_u^s(t) \leq 0
\]
Finally we say that network is \textbf{$p$-resource consistent} if under schedule $s$ probability of $u$ being resource consistent is at least $p$.
\subsection{The problem}
The problem that we are solving in this paper is determining whether TRN is consistent under acceptable risk in resource in time i.e. if acceptable risk in time is $r_T$ and acceptable risk in resource is $r_R$ we would like to find a schedule $s$ such that network is $(1-r_T)$-strongly controllable and $(1-r_R)$-resource consistent. Even though we frame the problem in this way it is not hard to alter the algorithm to consider constraint on sum $r_T + r_R$.
%%%%%%%%%%%%%%%%%%%%%%%%%%%%%% ALGORITHM %%%%%%%%%%%%%%%%%%%%%%%%%%%%%%%%%%%%%%
\section{Algorithm}
\blindtext[5]
%%%%%%%%%%%%%%%%%%%%%%%%%%%%%% EXPERIMENTS %%%%%%%%%%%%%%%%%%%%%%%%%%%%%%%%%%%%
\section{Experiments}
\blindtext[5]
%%%%%%%%%%%%%%%%%%%%%%%%%%%%%% FUTURE WORK %%%%%%%%%%%%%%%%%%%%%%%%%%%%%%%%%%%%
\section{Future Work}
\subsection{Conflict directed scheduling with resource limits(Peng and Szymon)}
For a network which there does not exist an assignment to controllable time points that is consistent with both time and resource constraints we wish to provide find a conflict - a suggested change to problem formulation which bring it closer to feasible solution.
\subsection{Possible extensions}
Below are possible extensions. I think the first one is simple to solve. The 2nd and 3rd are potentially hard. The last one sounds extremely hard to solve in reasonable execution time even for small problems.
\subsubsection{time-varying load profile}
Variable load profile can be approximated by a piecewise constant function with a series of temporal constraints and resource usage information. Does \textbf{not} require augmenting problem definition
\subsubsection{resource usage usage defined on random time points}
Requires augmenting problem definition.
\subsubsection{time-varying resource limit}
Could also potentially be approximated by piecewise constant function. Requires augmenting problem definition.
\subsubsection{stochastic resource limit}
Could potentially simulate multiple high likelihood scenarios to estimate probability of failure. Requires augmenting problem definition.%%%%%%%%%%%%%%%%%%%%%%%%%%%%%% CONCLUSION %%$$$%%%%%%%%%%%%%%%%%%%%%%%%%%%%%%%%
\section{Conclusion}
\blindtext[5]
%%%%%%%%%%%%%%%%%%%%%%%%%%%%%% REFERENCES %%%%%%%%%%%%%%%%%%%%%%%%%%%%%%%%%%%%%
\section{References}

\bibliography{references}
\end{multicols}
\end{document}
