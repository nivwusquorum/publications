\section{Problem statement}
In this section we will define notion of a Time Resource Network (TRN) and the relevant constraint on TRN's schedule - Resource Consistency. All the results presented in this paper can be extended to multiple different type of resources being constrained at the same time (electricity, water, fuel, cpu time, memory etc.), but to simplify the notation we will assume that only one type of resource is constrained. Additionally, for simplicity we only consider the problem of consistency, but the techniques presented in this paper can be easily extended to objective optimization over constrained schedules.
\subsection{Abstract Temporal Network}
Since TRNs can operate on top of many different types of temporal networks, we define a notion of Abstract Temporal Network (ATN), to capture only the necessary properties. For abstract temporal network we define two pieces of functionality:
\begin{enumerate}
\item \texttt{nodes(ATN)}, which returns a set of timepoints in $ATN$
\item \texttt{extend(ATN, $\{ stc_1, ... stc_n \} $)}, which takes ATN and a set of simple temporal constraints (\cite{dechter1991temporal}) spanning \texttt{nodes(ATN)}, and returns another $ATN'$, such that there exists a schedule satisfying $TC(ATN')$ if and only if there exists a schedule satisfying $TC(ATN)$ and the obeying set of simple temporal constraint $\{ stc_1, ... stc_n \} $. $TC$ is a notion of probabilistic temporal consistency described in section \ref{temporal_consistency}.
\end{enumerate}
As the following section describes in detail we will use \texttt{extend} to encode resource constraints over \texttt{nodes}.
\subsection{Schedule}
A schedule $s: \texttt{nodes(ATN)} \rightarrow \mathbb{R}$ is a mapping from abstract time points in ATN to concrete execution times.
\subsection{Temporal Consistency}
\label{temporal_consistency}
For an ATN we define a predicate $TC_s(ATN)$, which means that $ATN$ is \textbf{temporally consistent} under schedule $s$. $TC_s$ is true if schedule $s$ satisfies all the constraints of the $ATN$ (what that means precisely depends on the $ATN$ - we only require for it to be verifiable). We say that that $TC(ATN)$ which means that there exists at schedule $s$ such that $TC_s(ATN)$.
\paragraph{Example}
Example network that satisfies the ATN interface is Simple Temporal Network with Uncertainty (STNU) described in \cite{vidal1996dealing}. Let $N$ be an STNU. Using the terminology from the paper \texttt{nodes($N$)} is the set of received and activated nodes and \texttt{extend($N$, $\{ stc_1, ... stc_n \} $)} augments $N$ with $stc_i$ encoded as a controllable link. One way to define is $TC(N)$ is to be true if and only if $N$ is strongly controllable.
%Let's consider cc-pSTP \cite{Fang2014} as an example. Here \texttt{nodes} returns set of \textit{activated} and \textit{received} timepoints. \texttt{extend} returns network with extra \textit{free contraints} encoding the simple temporal constraints. The temporal consistency check $TC$ is true if cc-pSTP has a solution.
\subsection{Time Resource Network}
\label{sec:trn_definition}
A Time Resource Network is described by a tuple $TRN = (ATN, R)$, where $ATN$ is an Abstract Temporal Network and $R={src_1, ..., src_n}$ is a set of \textbf{simple resource constraints}, each of which is a triplet $(x, y, r)$, where $x, y \in$ \texttt{nodes(ATN)} and $r \in \mathbb{R}$ is resource usage which can be positive (consumption) and negative (generation). Given a schedule $s$ for any time $t \in \mathbb{R}$ resource \textbf{usage} for that simple resource constraint $src=(x,y,r)$ is equal to
\begin{align*}
u_s(src, t) = \begin{cases}
r & \text{if}\ s(x) \leq t < s(y)\\
0 & \text{otherwise}
\end{cases}
\end{align*}
Intuitively, simple resource constraint encodes the fact that between time $s(x)$ and $s(y)$  resource is consumed (generated) at the rate $|r|$ units of resource per unit time for positive (negative) $r$.

Our notation is inspired by \cite{bartusch1988scheduling}. In particular notice that one can encode arbitrary piecewise-constant resource profile, by decomposing it into simple resource constraint expressing usages at every constant intervals and simple temporal constraints joining their ends (details can be found in \cite{bartusch1988scheduling}).


\subsection{Resource consistency}
For a schedule $s$ we define a \textbf{net-usage} at time $t \in \mathbb{R}$ as $U_s(t)$ in the following way:
\[
U_s(t) = \sum_{\forall_{src_i \in R}} u_s(src_i, t)
\]
Where $R$ is a set of all the resource constraints and. We say that the network is \textbf{resource consistent} under schedule $s$ when it satisfies predicate $RC_s(TRN)$, i.e.
\begin{align}
\label{usage_for_all}\forall_{t \in {\mathbb{R} - C}} . U_s(t) \leq 0
\end{align}
where $C$ is some \textit{finite} set of real numbers. Intuitively it means that resource is never consumed at a rate which is greater than the generation rate. Set $C$ is introduced to make it easier to prove some properties, but is of no practical significance - notice that regardless of the contents of $C$ above statement is true 100 \% of time - there exists no finite interval where $U_s > 0$. We say that $TRN$ is resource consistent is there exists $s$ such that $RC_s(TRN)$.
\subsection{Time-resource consistency}
$TRN=(ATN, R)$ is \textbf{time-resource consistent} if there exists a schedule $s$ such that $RC_s(TRN) \wedge TC_s(ATN)$. Determining whether a $TRN$ is time-resource consistent is the central problem tackled in this publication.

\subsection{Properties of TRN}
Before we proceed to describe algorithms for determining time-resource consistency it will be helpful to understand some properties that apply to every TRN.
\begin{lemma}
\label{resource_checking}
For a $TRN$ a schedule $s$ is practically-resource-consistent if and only if
\begin{align}
\label{eq:resource_consistency}\forall_{t \in \texttt{nodes(ATN)}} \lim_{\epsilon \to 0} U_s(s(t) + \epsilon) \leq 0
\end{align}
i.e. resource usage is not non-positive a moment after all of the scheduled timepoints.
\end{lemma}
\begin{proof}
$\Rightarrow$ Trivial from definition of resource-consistency.
$\Leftarrow$ We say a time $t \in \mathbb{R}$ is scheduled if there exists a timepoint  $x \in \texttt{nodes(ATN)}$ such that $t = s(x)$. Assume that the right side of the implication is satisfied but the schedule is not resource consistent. That means that there exists a time point $t_{danger}$ for which $U_s(s(t_{danger})) > 0 $. We will only consider the case where $t_{danger}$ is \textbf{not} scheduled (because there are finitely many scheduled timepoints, we can consider them members of $C$). Let $t_{before}$ be the highest scheduled timepoint that is smaller than $t_{danger}$. Notice that if no such timepoint exist, this means that there's no resource constraint $(x,y,r)$ such that $s(x) \leq t_{danger} < s(y)$, so $U_s(s(t_{danger}))=0 $ . We can therefore assume that $t_{before}$ exists. Notice that by definition of $t_{before}$ and simple resource constraints, $U_s(t)$ for $t_{before} < t \leq t_{danger}$ is constant, therefore $U_s(s(t_{danger})) = \lim_{\epsilon \to 0} U_s(s(t_{before}) + \epsilon) > 0$. Contradiction.

\end{proof}
\begin{corollary}
\label{cor:ordering}
Given a $TRN$ with only simple resource constraints and two schedules $A$ and $B$ that have the same ordering of timepoints, $A$ is $p$-resource-consistent if and only if $B$ is $p$-resource-consistent.
\end{corollary}
\begin{proof}
Notice that if we move arbitrary timepoint, while preserving the relative ordering of timepoints, then net resource usage moment after that timepoint will not change (as the $U_s(t)$ between the neighboring timepoints remains constant). Therefore by lemma \ref{resource_checking} we can transform schedule $A$ into schedule $B$.
\end{proof}
