\section{Problem statement}
In this section we introduce a novel formulation - Time Resource Network (TRN). While the results presented in this paper can be extended to multiple types of resources being constrained simultaneously (for example electricity, water, fuel, CPU time and memory among others), for simplicity we consider only one type of constrained resource in this work. Additionally, we only consider the problem of consistency, but the techniques presented can be extended to handle optimization over constrained schedules.

\subsection{Abstract Temporal Network}
We wish to define TRN to support a general class of temporal networks. We thus define the notion of Abstract Temporal Network as a 3-tuple $ATN=\langle E, C, X\rangle$ where $E$ is a set of controllable events, $C$ is a set of simple temporal constraints \cite{dechter1991temporal} and $X$ represents any additional elements such as additional constraints and variables.

\paragraph{Schedule}
A schedule for an $ATN= \langle E,C,X \rangle$ is a mapping $s: \texttt{E} \rightarrow \mathbb{R}$ from events in ATN to their execution times.

\paragraph{Temporal Consistency}
\label{temporal_consistency}
For an $ATN=\langle E,C,X \rangle$ we define a predicate $TC_s(ATN) = stn-consistent(E,C,s) \wedge extra-criteria(E,C,X,s)$, which denotes the $ATN$ is \textbf{temporally consistent} under schedule $s$. $stn-consistent(E,C,s)$ represents STN consistency as defined in \cite{dechter1991temporal}.  $extra-criteria(E,C,X,s)$ depends on the type of the particular ATN. We say that $ATN$ is temporally consistent (denoted by $TC(ATN)$), if there exists at schedule $s$ such that $TC_s(ATN)$.

\paragraph{Example}
An example of a network that satisfies the ATN interface is Simple Temporal Network with Uncertainty (STNU) described in \cite{vidal1996dealing}. The set $E$ is composed of all the activated and received events, $C$ is the set of requirement links, $X$ is the set of all the contingent links. One way to define is $TC(ATN)$ is to be true if and only if the networks is strongly controllable (which already implies $stn-consistent(E,C,s)$).

%Let's consider cc-pSTP \cite{Fang2014} as an example. Here \texttt{events} returns set of \textit{activated} and \textit{received} time points. \texttt{extend} returns network with extra \textit{free contraints} encoding the simple temporal constraints. The temporal consistency check $TC$ is true if cc-pSTP has a solution.
\subsection{Time Resource Network}
\label{sec:trn_definition}
A Time Resource Network is described by a tuple $TRN = \langle ATN, R \rangle$, where $ATN$ is an Abstract Temporal Network and $R={src_1, ..., src_n}$ is a set of \textbf{simple resource constraints}, each of which is a triplet $ \langle x, y, r \rangle$, where $x, y \in$ E and $r \in \mathbb{R}$ is the amount of resource, which can be positive (consumption) and negative (generation). Given a schedule $s$ for any time $t \in \mathbb{R}$ we define \textbf{resource usage} for $src=\langle x,y,r \rangle$ as:
\begin{align*}
u_s(src, t) = \begin{cases}
r & \text{if}\ s(x) \leq t < s(y)\\
0 & \text{otherwise}
\end{cases}
\end{align*}
Intuitively, simple resource constraint encodes the fact that between time $s(x)$ and $s(y)$  resource is consumed (generated) at the rate $|r|$ per unit time for positive (negative) $r$.

Our notation is inspired by \cite{bartusch1988scheduling}. The authors have demonstrated that it is possible encode arbitrary piecewise-constant resource profile, by representing each constant interval by a simple resource constraint and joining ends of those intervals by simple temporal constraints.


\subsection{Resource consistency}
For a schedule $s$ we define a \textbf{net-usage} of a resource at time $t \in \mathbb{R}$ as:
\[
U_s(t) = \sum_{\forall_{src_i \in R}} u_s(src_i, t)
\]
$R$ is the set of all the resource constraints. We say that the network is \textbf{resource consistent} under schedule $s$ when it satisfies predicate $RC_s(TRN)$, i.e.
\begin{align}
\label{usage_for_all}\forall_{t \in \mathbb{R}} . U_s(t) \leq 0
\end{align}
Intuitively, it means that resource is never consumed at a rate that is greater than the generation rate. We say that $TRN$ is resource consistent, if there exists $s$, such that $RC_s(TRN)$ is true.

\subsection{Time-resource consistency}
$TRN=(ATN, R)$ is \textbf{time-resource consistent} if there exists a schedule $s$ such that $RC_s(TRN) \wedge TC_s(ATN)$. Determining whether a $TRN$ is time-resource consistent is the central problem addressed in this publication.

\subsection{Properties of TRN}
Before we proceed to describe algorithms for determining time-resource consistency it will be helpful to understand some properties common to every TRN.
\begin{lemma}
\label{resource_checking}
For a $TRN$ a schedule $s$ is resource consistent if and only if
\begin{align}
\label{eq:resource_consistency} \forall_{e \in E} U_s(s(e)) \leq 0
\end{align}
i.e. resource usage is non-positive a moment after all of the scheduled events.
\end{lemma}
\begin{proof}

$\Rightarrow$ Follows from definition of resource-consistency.\\
$\Leftarrow$ We say a time point $t \in \mathbb{R}$ is scheduled if there exists an event  $e \in E$ such that $t = s(e)$. Assume for contradiction, that the right side of the implication is satisfied, but the schedule is not resource consistent. That means that there exists a time point $t_{danger}$ for which $U_s(t_{danger}) > 0 $. Notice that by assumption $t_{danger}$ could not be scheduled. Let $t_{before}$ be the highest scheduled time point is smaller than $t_{danger}$. Notice that if no such time point existed, that would mean that there is no resource constraint $(x,y,r)$ such that $s(x) \leq t_{danger} < s(y)$, so $U_s(t_{danger})=0$ . By assumption, $U_s(t_{before}) < 0$.  We can therefore assume that $t_{before}$ exists. Notice that by definition of $t_{before}$ and simple resource constraints, $U_s(t)$ for $t_{before} \leq t \leq t_{danger}$ is constant. If it wasn't there would be another scheduled point between $t_{before}$ and $t_{danger}$, but we assumed that $t_{before}$ is highest scheduled point smaller than $t_{danger}$. Therefore $ U_s(t_{danger}) = U_s(t_{before})$. But we assumed that $U_s(t_{danger}) > 0$ and $U_s(t_{before}) < 0 $ Contradiction.
\end{proof}
\begin{corollary}
\label{cor:ordering}
Given a $TRN$ and two schedules $A$ and $B$ where all events occur in the same order, $A$ is resource consistent if and only if $B$ is resource consistent.
\end{corollary}
\begin{proof}
Notice that if we move execution time of arbitrary event, while preserving the relative ordering of all the events, then net resource usage at that event will not change. Therefore by lemma \ref{resource_checking},  $A$ is resource-consistent if and only if $B$ is resource-consistent.
\end{proof}
