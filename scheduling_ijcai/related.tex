\section{Related Work}
One of the earliest algorithmic solutions of a scheduling problem is mentioned in \cite{johnson1954optimal}, although the problem may have been considered in unpublished versions of \cite{bellman1956mathematical}. The scheduling problem considered in the publication featured $n$ items and $m$ stages, with $A_{i,j}$ denoting the time for $i$-th item to be processed by stage $j$. All items must be processed by the different stages in order (for example first stage is printing of a book and second stage is binding). The publication considered $m=2$ and $m=3$ and arrived at a solution that \textit{``permits one to optimally arrange twenty production items in about five minutes by visual inspection''}. Later work proved the solution to the problem for $m \geq 3$ is NP-hard (\cite{garey1976complexity}). In \cite{wagner1959integer} an integer programming solution to the scheduling problem was presented as \textit{``a single model which encompasses a wide variety of machine-scheduling situations''}.

In \cite{pritsker1969multiproject}, a generalization of scheduling problems with multiple resource constraints was considered. However, the proposed solution uses a discrete time formulation, which may prove intractable if higher resolution schedules were required. In 1988 a technique was proposed which can handle resource constraints and continuous time (\cite{bartusch1988scheduling}). The proposed approach can be thought of as resource constrained scheduling over Simple Temporal Networks (STN), a strict subset of the problems which can be modeled by the TRN.

In \cite{dechter1991temporal}, a notion of Simple Temporal Problem was introduced which allows one to solve problems with simple temporal constraints of form $l \leq t_y - t_x \leq u$. This concept was extended in \cite{vidal1996dealing}, which defined uncertain temporal constraints, where the duration between two time events can take a value from an interval $[l,u]$ unknown during the time of scheduling. A pseudopolynomial algorithm is described in \cite{morris2001dynamic} for finding scheduling policies based on the outcomes of observed uncertain durations (Dynamic controllability). The algorithm is later improved to polynomial complexity (\cite{morris2005temporal}). Finally, \cite{Fang2014} models problems where probabilistic information about the uncertain durations is known, and provides solutions based on a non-linear constraint programming encoding. The formulation of the TRN allows us to encode the temporal aspect of the temporal and resource constrained problem by building on the extensions as necessary. 
