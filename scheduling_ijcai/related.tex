\section{Related Work}
%\blindtext[5]

One of the earliest mentions of a scheduling problem being solved in an algorithmic fashion can be found in \cite{johnson1954optimal}, although there's evidence that the problem was already considered in unpublished versions of \cite{bellman1956mathematical}. This publication considers the following statement of scheduling problem. We have $n$ items and $m$ stages and $A_{i,j}$ denoting the time for $i$-th item to be processed by stage $j$. All the items must be processed by different stages in order (for example first stage is printing of a book and second stage is binding). The publication considers $m=2$ and $m=3$ and arrives at the solution that \textit{``permits one to optimally arrange twenty production items in about five minutes by visual inspection''}. It turns out that the solution to the problem for $m \geq 3$ is NP-hard (\cite{garey1976complexity}). In \cite{wagner1959integer} an Integer Programming solution to the scheduling problem and noticed that it \textit{``is a single model which encompasses a wide variety of machine-scheduling situations''}.

In \cite{pritsker1969multiproject} a generalization of scheduling problem is considered, which allows for multiple resource constraints. However the solution provided uses a discrete time formulation, which depending on required accuracy can substantially decrease performance. Work in this publication considers work on Temporal Networks, which explicitly model continuous time constraints. Interestingly, one of the publications about resource constrained scheduling (\cite{bartusch1988scheduling}) used the notion of which can be thought of as resource constrained scheduling over Simple Temporal Networks (STN). The publication derives the theory behind STNs 3 years before the STN publication!

In \cite{dechter1991temporal} a notion of Simple Temporal Problem was introduced which allows one to solve problem with simple temporal constraints of form $l \leq t_y - t_x \leq u$. This simple concept was later extended with various more sophisticated notions of temporal constraints. \cite{vidal1996dealing} defined the notion of uncertain temporal constraint, where the duration between two time events can take a value from an interval $[l,u]$ which is unknown during the time of scheduling (uncertain duration constraints); consistency of such temporal networks is called Strong Controllability. \cite{morris2001dynamic} desribes a pseudopolynamial algorithm for handling uncertain duration constraint, where we are allowed to make a descition scheduling decitions based on knowledge of uncertain durtions from the past (Dynamic controllability). His algorithm is later improved to polynamial complexity (\cite{morris2005temporal}). Finally, \cite{Fang2014} provides a non-linear optimization based solver for uncertain temporal constraints where the duration of the constraint can come from abritrary probabilistic distribution.
